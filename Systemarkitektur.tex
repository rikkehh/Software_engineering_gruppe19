\section{Systemarkitektur}
Systemarkitektur er en  anbefaling til systemets oppbygging og samspill mellom de ulike delene av systemet.

Tjenesten burde bli implementert i tråd med "Model-View-Controll" arkitektursmodellen, som da deler det logiske i tre deler. 
%hvor da tjenesten er delt opp i tre logiske deler som samhandler med hverandre. 
Grunnen er at det er standard i mange tjenester og har en del støtte til rammeverk og kodespråk. 

View korresponderer til brukergrensesnittet, altså det brukeren ser på skjermen sin. Det viser hvordan dataen presenterer til brukeren. 
Model er programmets datastruktur som administerer data, logikk og regler til tjenesten. Den får bruker data fra controller.
Conrtoller er mellomleddet mellom view og model. Den godtar input og konverterer de til kommandoer for model eller view. Logikken for hvordan dataen blir vist fram i brukergrensesnittet, og hvordan tjenesten responderer for ulike handlinger i view ligger her. 

Fordelen med å benytte den arkitekturen er at de er uavhengige av hverandre, og at hvis du endrer en av de komponentene, uten å påvirke de andre komponentene. 
Vi kan unngå replikasjon og kan bytte view for å tilfredstille andre platformer som mobilapplikasjon.

Tjenesten grensesnitt burde utvikles i samarbeid med UX UI designere. Dette er for å gjøre tjenesten forståelig og lett å bruke. Dette samsvarer 
med kravet om universell utforming, hvor alle skal kunne bruke tjenesten.
Ved å gjøre tjenesten lett og forståelig kan dette skape et større brukermasse. 




